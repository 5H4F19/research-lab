Evaluating teaching effectiveness is crucial for enhancing educational quality. Traditional evaluation methods, such as student surveys and peer reviews, often rely on subjective judgments and may not capture the full spectrum of teaching dynamics. These methods can be influenced by various biases and may overlook critical aspects of teaching performance.

Recent advancements in multimodal data analysis offer new opportunities to assess teaching effectiveness more comprehensively. By integrating visual, auditory, and textual data from classroom interactions, it is possible to gain deeper insights into teaching practices. However, several challenges persist in this approach. First, identifying and extracting relevant features from diverse data sources requires sophisticated techniques. Second, aligning and integrating these multimodal data streams to create a cohesive representation of teaching performance is complex. Third, developing models that can accurately interpret this integrated data to assess teaching effectiveness poses significant difficulties.

Addressing these challenges is essential for developing objective, scalable, and actionable methods for evaluating teaching effectiveness. Such advancements have the potential to inform instructional improvements and contribute to the overall enhancement of educational quality.
