It is important to evaluate teaching effectiveness in order to support the improvement of educational quality. Traditional evaluation approaches such as student surveys and peer evaluations tend to rely on subjective judgments, which implies the selection of certain facets of the classroom which may not encompass the full range of teaching effectiveness. Furthermore, their judgments may also be influenced by other biases or may overlook pertinent aspects of the classroom context regarding teaching effectiveness. Multimodal data analysis represents an emerging field of research that creates opportunities to evaluate teaching effectiveness more comprehensively. While it is true that multimodal data can provide a richer understanding of teaching effectiveness though visual, auditory, and text-based data from classroom interactions, there are still challenges to using multimodal data for this purpose. First, it is challenging to pinpoint and extract relevant features from the multiplicity of data streams. Second, it is difficult to align and bring together multimodal data to be utilized as a coherent representation of teaching effectiveness. Third, establishing models to evaluate multimodal data to identify teaching effectiveness poses a further level of complexity. Addressing challenges concerning multimodal data is essential if we are to develop objective, scalable and actionable methods to evaluate teaching effectiveness. Such methods might provide useful insights to improve instruction, and also contribute to enhancing quality in education overall.