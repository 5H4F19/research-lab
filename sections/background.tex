Assessing teaching effectiveness is a cornerstone of educational quality. Traditional evaluation methods, including student surveys, peer reviews, and classroom observations, are widely used due to their simplicity and ease of implementation. However, these methods are inherently subjective, prone to biases, and often fail to capture the complete range of instructional dynamics. Non-verbal communication, vocal intonation, and classroom interactions are aspects that are difficult to quantify with conventional approaches, yet they significantly influence learning outcomes.

Recent advancements in data science and machine learning have enabled the extraction and analysis of information from multiple modalities, such as video, audio, and textual data. Visual features can include gestures, facial expressions, and classroom engagement, while auditory features may capture voice tone, pitch, and speech rate. Textual features can be derived from lecture transcripts, student feedback, or teaching materials. Processing and integrating these heterogeneous data streams presents technical challenges, including feature extraction, temporal alignment, and multimodal fusion.

Several approaches have been proposed to leverage single modalities for educational assessment, yet the integration of multiple modalities remains underexplored. Multimodal learning analytics can provide a richer and more objective understanding of teaching practices, but it requires careful design of preprocessing pipelines, feature selection strategies, and interpretable predictive models. Logistic regression and other interpretable machine learning techniques offer a practical solution for modeling teaching effectiveness while maintaining transparency for educators and stakeholders.

By understanding the technical complexities involved in multimodal data processing, this research establishes a foundation for developing robust, objective, and scalable evaluation frameworks that can improve instructional practices and inform educational policy.
