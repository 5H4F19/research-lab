Evaluating teaching effectiveness plays a central role in ensuring educational quality. Traditional methods such as student surveys, peer reviews, and classroom observations are widely adopted due to their practical ease and cost-effectiveness. However, these conventional approaches suffer from inherent subjectivity, potential biases, and often fail to capture the full complexity of instructional dynamics. Non-verbal communication cues, vocal intonation, and interactive classroom behaviors critical factors influencing student learning are typically overlooked or difficult to quantify with standard evaluation tools.

Recent advancements in data science, machine learning, and sensor technologies have enabled the extraction and integrated analysis of information from multiple modalities, including video, audio, and textual data. Visual features, such as gestures, facial expressions, and spatial movement, provide insights into teacher presence and engagement. Auditory features encompass speech characteristics like tone, pitch, pace, and emotion, which reflect the nuances of verbal communication. Textual data, derived from lecture transcripts and classroom dialogue, reveal the structure, clarity, and cognitive complexity of instruction.

Leveraging these heterogeneous data streams presents technical challenges related to feature extraction, temporal alignment, and effective multimodal fusion. Despite the complexity, multimodal learning analytics promise richer and more objective assessments of teaching practices compared to single-modality methods. Existing applications have largely focused on unimodal approaches, yet integrating modalities can capture complementary behavioral and communicative cues that individually remain hidden.

Interpretable machine learning models such as logistic regression offer practical frameworks to combine multimodal features while maintaining transparency, which is crucial for educators and administrators to understand and trust evaluation outcomes. Developing robust multimodal assessment frameworks has the potential not only to improve instructional quality and student outcomes but also to inform educational policy and professional development with actionable data-driven insights.